\documentclass{beamer}
\usepackage[unilu,en]{collegeBeamer}
\usepackage[usenames,dvipsnames]{xcolor}
% Set to draft=true to show placeholders for missing images
\usepackage{tikz}
\usepackage{circuitikz}
\usepackage{movie15}
\usetikzlibrary{calc, arrows.meta, positioning, shapes.geometric, fit, backgrounds, decorations.pathreplacing}
\usepackage{listings}
\usepackage{fontawesome5}

% Code listing style
\lstset{
    basicstyle=\ttfamily\scriptsize,
    breaklines=true,
    frame=single,
    backgroundcolor=\color{gray!10}
}

% meta-data
\title{VAST Challenge 2022}
\subtitle{Patterns of Life in Engagement, Ohio\\Data Visualization}
\author{Alberto Finardi \\ Tommaso Crippa \\ Tom Gave}
\date{December 2025}
\themecolor{50,50,50}

% Image command that shows placeholder if image is missing
\usepackage{ifthen}
\let\oldincludegraphics\includegraphics
\renewcommand{\includegraphics}[2][]{%
    \IfFileExists{#2}{%
        \oldincludegraphics[#1]{#2}%
    }{%
        \fbox{\begin{minipage}[c][3cm][c]{4cm}%
            \centering\small\texttt{[Image]}\\[0.3cm]%
            \tiny\texttt{#2}%
        \end{minipage}}%
    }%
}

% document body
\begin{document}

\maketitle

% ==============================================================================
\section{Challenge Overview}
% ==============================================================================

\begin{frame}{The Problem}
    \vspace{15pt}
    \begin{columns}
        \begin{column}{0.5\textwidth}
            \textbf{Urban Planning Challenge}
            \begin{itemize}
                \item City of Engagement, Ohio
                \item Low knowledge of resident behavior
                \item Need data-driven insights
            \end{itemize}
        \end{column}
        \begin{column}{0.5\textwidth}
        \textbf{Challenge Scope}
            \begin{itemize}
                \item Map of urban area
                \item 15 months of data
                \item Diverse activity patterns
            \end{itemize}
        \end{column}
    \end{columns}
    \vspace{15pt}
    \begin{center}
        \textbf{Our Mission}
        \begin{columns}
        \begin{column}{0.5\textwidth}
        \begin{itemize}
            \item Analyze patterns of daily life
            \item Identify city characteristics
        \end{itemize}
        \end{column}
        \begin{column}{0.5\textwidth}
            \begin{itemize}
            \item Support infrastructure planning
            \item Improve quality of life
            \end{itemize}
        \end{column}
    \end{columns}
    \end{center}
    \vspace{0.5cm}
\end{frame}

\begin{frame}{The Dataset}
    \textbf{Massive Urban Activity Data}
    \begin{itemize}
        \item \textbf{Duration:} 15 months (March 2022 - May 2023)
        \item \textbf{Participants:} $\sim$1,000 volunteer residents
        \item \textbf{Data Volume:} $\sim$18GB of location and activity logs
        \item \textbf{Sampling Rate:} Every 5 minutes, 24/7
    \end{itemize}

    \vspace{15pt}
    \textbf{Data Sources}
    \begin{itemize}
        \item \textbf{Participant Status:} Location, activity mode, joviality
        \item \textbf{Buildings:} Venue types, locations, polygons
        \item \textbf{Travel Journal:} Trip origins, destinations, purposes
        \item and more...
    \end{itemize}

    \vspace{15pt}
    \textbf{Challenge:} Transform raw data into actionable urban insights
\end{frame}

\begin{frame}{Research Questions}
    \Large
    \begin{enumerate}
        \setlength{\itemsep}{20pt}
        \item \textbf{Question 1:} What are the distinct areas of the city?

        \item \textbf{Question 2:} Where are the traffic bottlenecks?

        \item \textbf{Question 3:} How do individual daily routines differ?

        \item \textbf{Question 4:} How do patterns change over time?
    \end{enumerate}
\end{frame}

% ==============================================================================
\section{Our Solution}
% ==============================================================================

\begin{frame}{Technology Stack}
    \begin{columns}
        \begin{column}{0.48\textwidth}
            \textbf{Frontend}
            \begin{itemize}
                \item React + TypeScript + Vite
                \item D3.js for interactive visualizations

            \end{itemize}

            \vspace{8pt}
            \textbf{Backend}
            \begin{itemize}
                \item Node.js + Express
                \item PostgreSQL with PostGIS
            \end{itemize}
        \end{column}

        \begin{column}{0.48\textwidth}
               \textbf{Deployment}
    \begin{itemize}
        \item Docker containerization for all services
                \item Nginx reverse proxy for API routing
        \item Optimized database with materialized views
    \end{itemize}
        \end{column}
    \end{columns}


\end{frame}

\begin{frame}{Architecture Overview}
    \centering
    \vspace{10pt}
    \begin{tikzpicture}[
        box/.style={rectangle, draw, thick, rounded corners=3pt, minimum height=1.2cm, minimum width=3cm, font=\small, align=center},
        container/.style={rectangle, draw=blue!60, line width=1.5pt, rounded corners=5pt, minimum height=5cm, minimum width=3.75cm, fill=blue!5},
        arrow/.style={->, >=Stealth, line width=1.2pt}
    ]
        % Frontend Container
        \node[container] (frontend_box) at (0,0) {};
        \node[font=\bfseries\normalsize, anchor=north, text=blue!70!black] at (frontend_box.north) [yshift=-0.3cm] {Frontend};
        \node[box, fill=orange!30] (react) at (0,0.3) {React + D3.js};
        \node[box, fill=orange!20] (nginx) at (0,-1.5) {Nginx};

        % Backend Container
        \node[container] (backend_box) at (4.5,0) {};
        \node[font=\bfseries\normalsize, anchor=north, text=blue!70!black] at (backend_box.north) [yshift=-0.3cm] {Backend};
        \node[box, fill=green!30, minimum height=3.2cm] (express) at (4.5,-0.7) {\raisebox{2cm}{Express.js}};
        \node[box, fill=green!20, minimum width=2cm] (processing) at (4.5,-1.5) {Data Processing};

        % Database Container
        \node[container] (db_box) at (9,0) {};
        \node[font=\bfseries\normalsize, anchor=north, text=blue!70!black] at (db_box.north) [yshift=-0.3cm] {Database};
        \node[box, fill=purple!30] (postgres) at (9,0.3) {PostgreSQL};
        \node[box, fill=purple!20] (postgis) at (9,-1.5) {PostGIS};

        % Arrows
        \draw[<->, >=Stealth, line width=0.8pt, gray] (nginx.north) -- (react.south);
        \draw[<->, >=Stealth, line width=0.8pt, gray] (postgis.north) -- (postgres.south);
        
        \draw[<->, >=Stealth, line width=1.2pt, black, dashed] (express.west) -- (nginx.east);
        \draw[<->, >=Stealth, line width=1.2pt, black, dashed] (express.east) -- (postgis.west);

        % Docker overlay
        \node[draw=gray!60, line width=2pt, rounded corners=8pt, fit={(frontend_box)(backend_box)(db_box)}, inner ysep=0.6cm, inner xsep=0.4cm, fill=none, dashed, yshift=-0.2cm] (docker) {};
        \node[anchor=north, font=\scriptsize\bfseries, text=gray!70] at (docker.south) [yshift=0.6cm] {\faDocker\hspace{0.2cm}Docker Compose};

    \end{tikzpicture}
\end{frame}

% ==============================================================================
\section{Visualization Techniques}
% ==============================================================================

% ========== Visualization 1: Spatial Heatmap ==========

\begin{frame}{}
    \centering
    \vfill
    {\Huge \textbf{How can we identify}}
    
    {\Huge \textbf{activity hotspots}}
    
    {\Huge \textbf{across the city?}}
    \vfill
\end{frame}

\begin{frame}{Visualization 1: Spatial Heatmap}
    \centering
    \vspace{15pt}
    %\includemovie{5cm}{5cm}{img/heatmap.mp4}
    \includegraphics[height=0.99\textheight,keepaspectratio]{img/morning_rush_hour_activity.png}
\end{frame}


\begin{frame}{Spatial Heatmap - Purpose \& Features}
    \begin{columns}
        \begin{column}{0.48\textwidth}
            \textbf{Purpose}
            \begin{itemize}
                \item Visualize activity density across the city
                \item Identify busy areas and hotspots
                \item Track temporal patterns
            \end{itemize}

            \vspace{15pt}
            \textbf{Key Features}
            \begin{itemize}
                \item Grid-based aggregation
                \item Time slider (hourly/daily/weekly)
                \item Interest Group filtering
                \item Building polygon overlay
                \item Interactive zoom and pan
            \end{itemize}
        \end{column}
        \begin{column}{0.48\textwidth}
            \centering
            \includegraphics[width=\textwidth,keepaspectratio]{img/interest_group_homogeneity.png}
        \end{column}
    \end{columns}
\end{frame}

\begin{frame}{Spatial Heatmap - Evaluation}
    \begin{columns}
        \begin{column}{0.48\textwidth}
            \textbf{Pros}
            \begin{itemize}
                \setlength{\itemsep}{8pt}
                \item Intuitive geographic representation
                \item Reveals spatial patterns at a glance
                \item Flexible temporal exploration
                \item Supports multiple aggregation levels
                \item Combines well with building overlays
                \item Effective for identifying hotspots
            \end{itemize}
        \end{column}
        \begin{column}{0.48\textwidth}
            \textbf{Cons}
            \begin{itemize}
                \setlength{\itemsep}{8pt}
                \item Grid resolution affects interpretation
                \item Can obscure individual movements
                \item Performance challenges with high granularity
                \item Requires spatial context to interpret
                \item May hide temporal variations within aggregates
            \end{itemize}
        \end{column}
    \end{columns}
\end{frame}

% ========== Visualization 2: Activity Streamgraph ==========

\begin{frame}{}
    \centering
    \vfill
    {\Huge \textbf{How do activity patterns}}
    
    {\Huge \textbf{evolve over time?}}
    \vfill
\end{frame}

\begin{frame}{Visualization 2: Activity Streamgraph}
    \centering
    \includegraphics[width=0.99\textwidth,keepaspectratio]{img/activity_streamgraph.png}
\end{frame}


\begin{frame}{Activity Streamgraph - Purpose \& Features}
    \begin{columns}
        \begin{column}{0.48\textwidth}
            \textbf{Purpose}
            \begin{itemize}
                \item Show activity percentage over time
                \item Reveal behavioral shifts
                \item Track participation trends
            \end{itemize}
        \end{column}
        \begin{column}{0.48\textwidth}
            \textbf{Key Features}
            \begin{itemize}
                \item Stacked area chart with smooth interpolation
                \item Multiple activity types (work, social, etc.)
                \item Temporal filtering capabilities
                \item Color-coded activity categories
            \end{itemize}
        \end{column}
    \end{columns}
    \vspace{15pt}
                \centering
            \includegraphics[height=0.40\textheight,keepaspectratio]{img/18_diverse_activity.png}
\end{frame}

\begin{frame}{Activity Streamgraph - Evaluation}
    \begin{columns}
        \begin{column}{0.48\textwidth}
            \textbf{Pros}
            \begin{itemize}
                \setlength{\itemsep}{8pt}
                \item Shows composition and trends simultaneously
                \item Aesthetically appealing and engaging
                \item Reveals both macro and micro patterns
                \item Effective for time-series comparison
                \item Handles multiple categories elegantly
            \end{itemize}
        \end{column}
        \begin{column}{0.48\textwidth}
            \textbf{Cons}
            \begin{itemize}
                \setlength{\itemsep}{8pt}
                \item Difficult to read precise values
                \item Middle layers harder to interpret
                \item Can be overwhelming with too many categories
                \item Requires color differentiation
                \item Temporal aggregation may hide short-term spikes
            \end{itemize}
        \end{column}
    \end{columns}
\end{frame}

% ========== Visualization 3: Activity Calendar ==========

\begin{frame}{}
    \centering
    \vfill
    {\Huge \textbf{How do individual}}
    
    {\Huge \textbf{daily routines differ?}}
    \vfill
\end{frame}

\begin{frame}{Visualization 3: Activity Calendar}
    \centering
    \vspace{10pt}
    \includegraphics[height=0.99\textheight,keepaspectratio]{img/participant_845_activity_calendar.png}
\end{frame}

\begin{frame}{Activity Calendar - Purpose \& Features}
    \begin{columns}
        \begin{column}{0.48\textwidth}
            \textbf{Purpose}
            \begin{itemize}
                \item Analyze individual daily routines
                \item Identify patterns and variations
                \item Compare participants side-by-side
            \end{itemize}

            \vspace{15pt}
            \textbf{Key Features}
            \begin{itemize}
                \item Days $\times$ Hours matrix
                \item Color-coded by activity type
                \item One month visible at a glance
                \item Scrollable timeline for full 15-month period
            \end{itemize}
        \end{column}
        \begin{column}{0.48\textwidth}
            \centering
            \includegraphics[width=0.80\textwidth,keepaspectratio]{img/participant_744_calendar.png}
        \end{column}
    \end{columns}
\end{frame}

\begin{frame}{Activity Calendar - Evaluation}
    \begin{columns}
        \begin{column}{0.48\textwidth}
            \textbf{Pros}
            \begin{itemize}
                \setlength{\itemsep}{8pt}
                \item Compact representation of long periods
                \item Patterns emerge naturally (work hours, weekends)
                \item Easy to spot anomalies and changes
                \item Effective for individual analysis
                \item Supports direct comparison
                \item Intuitive time-of-day interpretation
            \end{itemize}
        \end{column}
        \begin{column}{0.48\textwidth}
            \textbf{Cons}
            \begin{itemize}
                \setlength{\itemsep}{8pt}
                \item Limited to individual
                \item Requires significant screen space
                \item Can be cluttered with too many activity types
                \item Doesn't show spatial information
                \item Difficult to see population-level trends
            \end{itemize}
        \end{column}
    \end{columns}
\end{frame}

% ========== Visualization 4: Building Polygons Overlay ==========

\begin{frame}{}
    \centering
    \vfill
    {\Huge \textbf{How can we understand}}
    
    {\Huge \textbf{urban infrastructure context?}}
    \vfill
\end{frame}

\begin{frame}{Visualization 4: Building Polygons Overlay}
    \centering
    \vspace{15pt}
    \includegraphics[height=0.99\textheight,keepaspectratio]{img/engagement.png}
\end{frame}

\begin{frame}{Building Polygons - Purpose \& Features}
    \begin{columns}
        \begin{column}{0.48\textwidth}
            \textbf{Purpose}
            \begin{itemize}
                \item Provide spatial context for activity patterns
                \item Link activities to physical infrastructure
                \item Identify functional zones
            \end{itemize}

            \vspace{15pt}
            \textbf{Key Features}
            \begin{itemize}
                \item Filter by building type
                \item Integrated with heatmap for layered context
                \item Color-coded by function
            \end{itemize}
        \end{column}
        \begin{column}{0.48\textwidth}
            \centering
            \includegraphics[width=\textwidth,keepaspectratio]{img/residential_buildings_map.png}
        \end{column}
    \end{columns}
\end{frame}

\begin{frame}{Building Polygons - Evaluation}
    \begin{columns}
        \begin{column}{0.48\textwidth}
            \textbf{Pros}
            \begin{itemize}
                \setlength{\itemsep}{8pt}
                \item Connects activity to infrastructure
                \item Helps explain spatial patterns
                \item Supports urban planning decisions
                \item Reveals functional zoning
                \item Combines well with other visualizations
            \end{itemize}
        \end{column}
        \begin{column}{0.48\textwidth}
            \textbf{Cons}
            \begin{itemize}
                \setlength{\itemsep}{8pt}
                \item Can clutter the map
                \item Requires accurate building data
                \item May obscure underlying heatmap
                \item Static representation of dynamic spaces
            \end{itemize}
        \end{column}
    \end{columns}
\end{frame}

% ========== Visualization 5: Participant Comparison ==========

\begin{frame}{}
    \centering
    \vfill
    {\Huge \textbf{How can we compare}}
    
    {\Huge \textbf{individual lifestyles?}}
    \vfill
\end{frame}

\begin{frame}{Visualization 5: Participant Comparison}
    \centering
    \vspace{15pt}
    \includegraphics[width=0.99\textwidth,keepaspectratio]{img/participant_comparison.png}
\end{frame}


\begin{frame}{Participant Comparison - Purpose \& Features}
    \begin{columns}
        \begin{column}{0.48\textwidth}
            \textbf{Purpose}
            \begin{itemize}
                \item Compare individual behavioral patterns
                \item Identify contrasting lifestyles
                \item Support hypothesis about proximity and well-being
            \end{itemize}
        \end{column}
        \begin{column}{0.48\textwidth}
                        \textbf{Key Features}
            \begin{itemize}
                \item Daily travel distance
                \item Average joviality score
                \item Social activity percentage
                \item Work patterns
                \item Demographics
            \end{itemize}
        \end{column}
    \end{columns}
\end{frame}

\begin{frame}{Participant Comparison - Evaluation}
    \begin{columns}
        \begin{column}{0.48\textwidth}
            \textbf{Pros}
            \begin{itemize}
                \setlength{\itemsep}{8pt}
                \item Direct quantitative comparison
                \item Reveals individual differences
                \item Supports finding extreme cases
                \item Evidence-based storytelling

            \end{itemize}
        \end{column}
        \begin{column}{0.48\textwidth}
            \textbf{Cons}
            \begin{itemize}
                \setlength{\itemsep}{8pt}
                \item Limited to 2 participants at once
                \item Doesn't show population distribution
                \item Risk of cherry-picking examples
                \item Requires manual selection
            \end{itemize}
        \end{column}
    \end{columns}
\end{frame}

\begin{frame}{Data Aggregation Strategy}
    \textbf{Spatial Aggregation}
    \begin{itemize}
        \item Grid-based binning (configurable cell size)
        \item \texttt{GROUP BY FLOOR(lat/cell\_size), FLOOR(lng/cell\_size)}
        \item Enables density calculation and hotspot identification
    \end{itemize}

    \vspace{8pt}
    \textbf{Temporal Aggregation}
    \begin{itemize}
        \item Hourly: Daily patterns and rush hours
        \item Daily/Weekly: Routine identification
        \item Monthly: Long-term trends
    \end{itemize}

    \vspace{8pt}
    \textbf{Activity Mode Filtering}
    \begin{itemize}
        \item Work, Home, Restaurant, Pub, Recreation, School, Shopping
        \item Supports focused analysis by activity type
        \item Reveals functional zones in the city
    \end{itemize}
\end{frame}

\section{Reflection}


\begin{frame}{Strengths \& Limitations}
    \begin{columns}
        \begin{column}{0.48\textwidth}
            \textbf{Strengths}
            \begin{itemize}
                \item Interactive exploration
                \item Multi-scale analysis
                \item Evidence-based insights
                \item Scalable architecture
                \item Accessible visualizations
            \end{itemize}

            \vspace{10pt}
            \textbf{Future Enhancements}
            \begin{itemize}
                \item Flow diagrams
                \item Speed heatmaps
                \item Predictive modeling
                \item Real-time integration
            \end{itemize}
        \end{column}

        \begin{column}{0.48\textwidth}
            \textbf{Limitations}
            \begin{itemize}
                \item Sample representativeness
                \item No causal analysis
                \item Performance constraints
                \item Learning curve
            \end{itemize}

            \vspace{10pt}
            \textbf{Applicability}
            \begin{itemize}
                \item Urban planning
                \item Transportation analysis
                \item Behavioral studies
                \item Decision support
            \end{itemize}
        \end{column}
    \end{columns}
\end{frame}

\begin{frame}{Key Takeaways}
    \Large
    \begin{enumerate}
        \setlength{\itemsep}{20pt}
        \item \textbf{PostGIS essential} for spatial data

        \item \textbf{Iterative design} reveals insights

        \item \textbf{Performance matters} for 18GB datasets

        \item \textbf{Context drives interpretation}

        \item \textbf{Visual analytics} enables discovery
    \end{enumerate}
\end{frame}

% ==============================================================================
\section{Conclusion}
% ==============================================================================

\begin{frame}{Summary: Challenge Accomplished}
    \Large
    \textbf{Answered All Four Questions}

    \vspace{15pt}
    \normalsize
    \begin{enumerate}
        \setlength{\itemsep}{12pt}
        \item \textbf{City Areas:} 3 distinct zones identified

        \item \textbf{Bottlenecks:} 3 critical congestion points

        \item \textbf{Routines:} 6× difference in commute impacts life

        \item \textbf{Changes:} 10 temporal patterns documented
    \end{enumerate}

    \vspace{20pt}
    \Large
    \textbf{Delivered actionable urban planning insights}

    \textbf{through interactive visual analytics}
\end{frame}

\begin{frame}{}
    \centering
    \vspace{2cm}
    {\Huge \textbf{Thank You!}}

    \vspace{1.5cm}
    {\large Questions?}

    \vspace{1.5cm}
    \begin{tabular}{ccc}
        Alberto Finardi & Tommaso Crippa & Tom Gave \\
    \end{tabular}

    \vspace{0.8cm}
    \scriptsize
    Data Visualization -- University of Luxembourg

    \vspace{0.5cm}
    {\tiny VAST Challenge 2022: Challenge 2 - Patterns of Life}
\end{frame}

\end{document}
